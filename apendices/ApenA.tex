\chapter{Esta plantilla est� orientada a Inform�tica}\label{apenA}
Los c�digos de programas que se utilizaron en la creaci�n o desarrollo de la tesis se incluyen utilizando la siguiente forma:

El C�digo \ref{cod:gauss} muestra el programa que se utiliz� para la adquisici�n de datos desde el coraz�n vivo del paciente.
\begin{mdframed}[linecolor=black, topline=false, bottomline=false, leftline=false, rightline=false, userdefinedwidth=\textwidth]
\captionof{lstlisting}{El c�digo en C de la funci�n de membres�a Gaussiana.}
\label{cod:gauss}
\vspace{-0.3cm} %separci�n entre el caption y el c�digo
\begin{minted}[mathescape, breaklines, frame=single, tabsize=4, gobble=0, linenos, numbersep=5pt,]{c}
////////////////////////////////////////////////
// Funci�n de membres�a Guassiana 
//$f(x)=e^{-\alpha \frac{x - c}{\delta}^2}$
////////////////////////////////////////////////

double gaussiana(double centro, double ancho, double x){
	double ux=0;
	ux=pow(2.718281828, ( -.5* pow(((x-centro) / ancho), 2)));
	return ux;
}
\end{minted}
\end{mdframed}

\begin{mdframed}[linecolor=black, topline=false, bottomline=false, leftline=false, rightline=false, userdefinedwidth=\textwidth]
\captionof{lstlisting}{El c�digo en C de la funci�n de membres�a Hombro Izquierdo.}
\label{cod:funhombizquiero}
\vspace{-0.3cm} %separci�n entre el caption y el c�digo
\begin{minted}[mathescape,breaklines, frame=single, tabsize=4, gobble=0, linenos, numbersep=5pt,]{c}
double hombroIzquierdo(double a, double b, double x){
	double ux=0.0;
	if(x<=a){
		ux=1;
	}
	if(x>a && x<b){
		ux=(b-x)/(b-a);
	}
	if(x>b){
		ux=0;
	}
	return ux;
}
\end{minted}
\end{mdframed}